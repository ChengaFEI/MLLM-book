\documentclass{book}
\usepackage{hyperref}

\begin{document}

\frontmatter
\title{Instruction Pruning and Prompt Engineering for Multimodal Large Language Models}
\author{Your Name}
\maketitle
\tableofcontents

\mainmatter

\chapter{Introduction to Instruction Pruning and Prompt Engineering in MLLM}

\section{Overview of MLLMs and LLMs}
\section{The Importance of Efficiency and Adaptability}
\section{Challenges in Model Performance and Bias Mitigation}

\chapter{Instruction Pruning Techniques}
\section{Dynamic Token Pruning}
\subsection{Overview of Dynamic Token Pruning}
Dynamic token pruning, exemplified by the innovative LazyLLM approach introduced by Fu et al. (2024), is a cutting-edge technique designed to enhance the efficiency of large language models (LLMs) during inference, particularly in long-context scenarios. This method addresses performance bottlenecks in the prefilling stage by dynamically selecting and processing subsets of tokens, selectively computing the Key-Value (KV) cache, and reincorporating previously pruned tokens. While primarily focused on text-based LLMs, the concept shows promise for extension to multimodal contexts, such as optimizing image token processing in Large Multimodal Models (LMMs) and mitigating language biases in multilingual vision-language models. The adoption of dynamic token pruning techniques like LazyLLM offers significant benefits, including improved efficiency (demonstrated by a 2.34 times acceleration in the prefilling stage for the LLama 2 7B model), maintained accuracy despite reduced computational load, and potential adaptability to various LLM architectures and tasks, making it a versatile solution for existing and future language models across unimodal and multimodal applications.

As research in this area progresses, dynamic token pruning is likely to play a crucial role in enhancing the efficiency and adaptability of both unimodal and multimodal large language models.

\subsection{LazyLLM Approach}
\textit{Example:} Implementing LazyLLM for multi-document QA tasks \cite{fu2024}

\section{Coarse-to-Fine Pruning Strategies}
\subsection{CoT-Influx for Mathematical Reasoning}
\textit{Example:} Applying CoT-Influx to enhance few-shot learning in math problems \cite{huang2023}

\chapter{Advanced Prompt Engineering Methods}
\section{Bias Mitigation in Prompts}
\subsection{Reducing Gender Bias in Machine Translation}
\textit{Example:} Structuring prompts to minimize gender bias in LLM translations \cite{sant2024}

\section{Memorization and Security Considerations}
\subsection{Uncovering Data Leakage through Instruction-Based Prompts}
\textit{Example:} Using Alpaca against Vicuna to expose pre-training data \cite{kassem2024}

\subsection{Protecting Against Malicious Prompts}
\textit{Example:} Implementing safeguards against MaPP attacks in code generation \cite{heibel2024}

\chapter{Efficiency and Adaptability in LLMs}
\section{Mixture of Experts (MoE) Approaches}
\subsection{Training-Free MoE for Sequence-Level Expert Selection}
\textit{Example:} Implementing GRIFFIN for efficient model deployment \cite{dong2024}

\section{Interactive Learning and Environment Adaptation}
\subsection{Autonomous LLM Agents for New Environments}
\textit{Example:} Designing an AutoManual framework for task adaptation \cite{chen2024}

\chapter{Practical Applications and Case Studies}
\section{Multi-Document Question Answering}
\section{Mathematical Reasoning and Problem Solving}
\section{Cross-Lingual Tasks and Translation}
\section{Code Generation and Security}

\chapter{Challenges and Future Directions}
\section{Balancing Efficiency and Performance}
\section{Addressing Bias and Fairness}
\section{Enhancing Security and Privacy}
\section{Improving Adaptability to New Domains}

\chapter{Conclusion}
\section{Summary of Key Findings}
\section{Implications for AI Research and Development}
\section{Future Research Opportunities}

\backmatter
\bibliographystyle{plain}
\bibliography{references}

\end{document}
